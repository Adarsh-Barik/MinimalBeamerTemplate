%% References: http://users.eecs.northwestern.edu/~jesse/code/beamer-examples/
%% https://github.com/matze/mtheme/
%% Set beamer template but no theme
\documentclass[xetex]{beamer}

%% Some essential packages
\usepackage{amsmath, amssymb, amsfonts, amsthm}
\usepackage[quiet]{fontspec}
\usepackage{xunicode}
\usepackage{xltxtra}
\usepackage{graphicx}
\usepackage{stmaryrd}
\usepackage{xcolor}
\usepackage{tikz}
\usepackage{booktabs}
\usepackage{url}

%% Some non-essential packages 
\usepackage{appendixnumberbeamer}
\usepackage{pgfplots}


%% SETTINGS

\frenchspacing
\unitlength=0.01\textwidth
\thicklines
\urlstyle{sf}
\graphicspath{{images/}}

%% Colors : use capital letters because xcolor has problems reading f
%%  Pallet from https://elementary.io/brand
\def\blueberry{3892A0}
\def\strawberry{DA4D45}
\def\morange{F37329}
\def\banana{FBD25D}
\def\lime{93D844}
\def\grape{8A4EBF}
\def\mwhite{FFFFFF}
\def\mblack{333333}
\def\soothingwhite{FAFCF8}

\definecolor{Blueberry}{HTML}{\blueberry}
\definecolor{Strawberry}{HTML}{\strawberry}
\definecolor{MOrange}{HTML}{\morange}
\definecolor{Banana}{HTML}{\banana}
\definecolor{Lime}{HTML}{\lime}
\definecolor{Grape}{HTML}{\grape}
\definecolor{MWhite}{HTML}{\mwhite}
\definecolor{MBlack}{HTML}{\mblack}
\definecolor{SWhite}{HTML}{\soothingwhite}

%% Set colors
\setbeamercolor{title}{fg=Blueberry}
\setbeamercolor{frametitle}{fg=Blueberry}
\setbeamercolor{normal text}{fg=MBlack}

\setbeamercolor{block title}{fg=Blueberry}
\setbeamercolor{block body}{fg=MBlack}
\setbeamercolor{block title alerted}{fg=Strawberry}
\setbeamercolor{block title example}{fg=Lime}

\setbeamercolor{alerted text}{fg=Strawberry}
\setbeamercolor{itemize item}{fg=MBlack}
\setbeamercolor{enumerate item}{fg=MBlack}
\setbeamercolor{description item}{fg=Grape}

\setbeamercolor{section in toc}{fg=MBlack}


\setbeamercolor{background canvas}{
        bg=SWhite
 }


%% Set fonts
\defaultfontfeatures{
    Mapping=tex-text,
    Scale=MatchLowercase,
}
\setsansfont{Optima}
\setmonofont{Monaco}

%% Set templates for various elements

\setbeamertemplate{frametitle}
  {\begin{centering}\smallskip
   \insertframetitle\par
   \smallskip\end{centering}}

\setbeamertemplate{itemize item}{$\bullet$}

\setbeamertemplate{navigation symbols}{}
\setbeamertemplate{footline}[text line]{%
    \hfill\strut{%
        \scriptsize\sf\color{black!60}%
        \quad\insertframenumber%
    }%
    \hfill%
}



%% Title and authors
\title{Demo slide in Beamer}
\subtitle{A minimal theme}
\date{\today}
\author{\underline{Adarsh Barik}}
\institute{Purdue University}


%% DOCUMENT BEGINS HERE
\begin{document}

\maketitle

%% Toc in first frame - takes sections and not frames
\begin{frame}{Table of contents}
  \setbeamertemplate{section in toc}[sections numbered]
  \tableofcontents[hideallsubsections]
\end{frame}

%% Just for reference - it doesn't create a new slide
\section{Introduction}

\begin{frame}[fragile]{My theme}

  The  theme is a Beamer template with minimal visual noise inspired by template by Jesse A. Tov and Metropolis theme.


  Note, that you have to have Optima and Monaco font and XeTeX
  installed to enjoy this wonderful typography. This theme has been compiled using:
  \begin{verbatim}
  xelatex -halt-on-error -interaction=nonstopmode demo
  \end{verbatim} 
\end{frame}

\begin{frame}[fragile]{Sections}
  Sections group slides of the same topic

  \begin{verbatim}    \section{Elements}\end{verbatim}


\end{frame}


\section{Elements}

\begin{frame}[fragile]{Typography}
      \begin{verbatim}The theme provides sensible defaults to
\emph{emphasize} text, \alert{accent} parts
or show \textbf{bold} results.\end{verbatim}

  \begin{center}becomes\end{center}

  The theme provides sensible defaults to \emph{emphasize} text,
  \alert{accent} parts or show \textbf{bold} results.
\end{frame}

\begin{frame}{Font feature test}
  \begin{itemize}
    \item Regular
    \item \textit{Italic}
    \item \textsc{SmallCaps}
    \item \textbf{Bold}
    \item \textbf{\textit{Bold Italic}}
    \item \textbf{\textsc{Bold SmallCaps}}
    \item \texttt{Monospace}
    \item \texttt{\textit{Monospace Italic}}
    \item \texttt{\textbf{Monospace Bold}}
    \item \texttt{\textbf{\textit{Monospace Bold Italic}}}
  \end{itemize}
\end{frame}

\begin{frame}{Lists}
  \begin{columns}[T,onlytextwidth]
    \column{0.33\textwidth}
      Items
      \begin{itemize}
        \item Milk \item Eggs \item Potatos
      \end{itemize}

    \column{0.33\textwidth}
      Enumerations
      \begin{enumerate}
        \item First, \item Second and \item Last.
      \end{enumerate}

    \column{0.33\textwidth}
      Descriptions
      \begin{description}
        \item[PowerPoint] Meeh. \item[Beamer] Yeeeha.
      \end{description}
  \end{columns}
\end{frame}
\begin{frame}{Animation}
  \begin{itemize}[<+- | alert@+>]
    \item \alert<4>{This is\only<4>{ really} important}
    \item Now this
    \item And now this
  \end{itemize}
\end{frame}
\begin{frame}{Figures}
  \begin{figure}
    \newcounter{density}
    \setcounter{density}{20}
    \begin{tikzpicture}
      \def\couleur{alerted text.fg}
      \path[coordinate] (0,0)  coordinate(A)
                  ++( 90:5cm) coordinate(B)
                  ++(0:5cm) coordinate(C)
                  ++(-90:5cm) coordinate(D);
      \draw[fill=\couleur!\thedensity] (A) -- (B) -- (C) --(D) -- cycle;
      \foreach \x in {1,...,40}{%
          \pgfmathsetcounter{density}{\thedensity+20}
          \setcounter{density}{\thedensity}
          \path[coordinate] coordinate(X) at (A){};
          \path[coordinate] (A) -- (B) coordinate[pos=.10](A)
                              -- (C) coordinate[pos=.10](B)
                              -- (D) coordinate[pos=.10](C)
                              -- (X) coordinate[pos=.10](D);
          \draw[fill=\couleur!\thedensity] (A)--(B)--(C)-- (D) -- cycle;
      }
    \end{tikzpicture}
    \caption{Rotated square from
    \href{http://www.texample.net/tikz/examples/rotated-polygons/}{texample.net}.}
  \end{figure}
\end{frame}
\begin{frame}{Tables}
  \begin{table}
    \caption{Largest cities in the world (source: Wikipedia)}
    \begin{tabular}{@{} lr @{}}
      \toprule
      City & Population\\
      \midrule
      Mexico City & 20,116,842\\
      Shanghai & 19,210,000\\
      Peking & 15,796,450\\
      Istanbul & 14,160,467\\
      \bottomrule
    \end{tabular}
  \end{table}
\end{frame}
\begin{frame}{Blocks}
  Three different block environments are pre-defined and may be styled with an
  optional background color.

  \begin{columns}[T,onlytextwidth]
    \column{0.5\textwidth}
      \begin{block}{Default}
        Block content.
      \end{block}

      \begin{alertblock}{Alert}
        Block content.
      \end{alertblock}

      \begin{exampleblock}{Example}
        Block content.
      \end{exampleblock}

    \column{0.5\textwidth}

      %\metroset{block=fill}

      \begin{block}{Default}
        Block content.
      \end{block}

      \begin{alertblock}{Alert}
        Block content.
      \end{alertblock}

      \begin{exampleblock}{Example}
        Block content.
      \end{exampleblock}

  \end{columns}
\end{frame}
\begin{frame}{Math}
  \begin{equation*}
    e = \lim_{n\to \infty} \left(1 + \frac{1}{n}\right)^n
  \end{equation*}
\end{frame}



\begin{frame}{References}
  Some references to showcase \cite{knuth92,ConcreteMath,Simpson,Er01,greenwade93}
\end{frame}

\section{Conclusion}

\begin{frame}{Summary}

  Get the source of this template and the demo presentation from

  \begin{center}\url{github.com/Adarsh-Barik}\end{center}



\end{frame}

\begin{frame}
  \centering Questions?
\end{frame}

%\appendix

\begin{frame}[fragile]{Backup slides}
  Sometimes, it is useful to add slides at the end of your presentation to
  refer to during audience questions.

\end{frame}

\begin{frame}[allowframebreaks]{References}

  \bibliographystyle{apalike}
  \bibliography{demo}

\end{frame}

\end{document}